%==========================================================
%======================  FRONT STUFF ======================
%==========================================================

%======================= Cover Page =======================
%\begin{document}
\newpage
\pagestyle{plain}
\pagenumbering{roman}
\setcounter{page}{1}
\addcontentsline{toc}{section}{Title Page}
%\setlength{\footskip}{-15mm}
\setlength{\textheight}{230mm}
\setlength{\voffset}{10mm}
\vskip -5em

\begin{center}
{ 
  \singlespace \uppercase{\bf Understanding the Argumentation Structure of Marine Insurance Legal Disputes
} \par
}
\vskip 2em
{
  \lineskip 1.5em
  \begin{tabular}[t]{c} by\\ \\ Nischal Srinivas Basuti
  \end{tabular}\par
}
%\vskip 5.6em

\vskip 3.5em
\singlespace A research study submitted in partial fulfillment of the
requirements for the\\ degree of Master of Engineering in\\ 
Computer Science 

%	\vskip 5em
\vskip 0.5em
{
  \singlespace
  \begin{center}
    \begin{tabular}{rl}
      \\[-1em]
      Examination Committee: & Prof.\ Phan Minh Dung (Chairperson) \\[-0.8em]
                             & Prof.\ Matthew N. Dailey \\[-0.8em]
                             & Dr.\ Attaphongse Taparugssanagorn \\\\

% UNCOMMENT THE LINES BELOW IF YOU HAVE THE EXTERNAL EXAMINER.
%      External Examiner:     & Prof.\ YOUR EXTERNAL EXAMINER \\[-0.8em]
%                             & Dept.\ of Electrical and Computer Engineering \\[-0.8em]
%                             & McGill University, Canada \\\\
		
      Nationality:     & Indian \\[-0.8em]
      Previous Degree: & Bachelor of Technology in Computer Science\\[-0.8em]
                       & Jawaharlal Nehru Technological University Hyderabad,\\[-0.8em]
                       & Telangana, India\\[-0.8em]

      \\
      Scholarship Donor: & AIT Fellowship\\[-0.8em]
      \\
    \end{tabular}
  \end{center}

  \vskip 3.0em
  \centerline{}
  \vskip 2em
}
\end{center}
\begin{center}
  \singlespace Asian Institute of Technology\\ School of Engineering
                  and Technology\\ Thailand\\ December 2020
\end{center}
\vfill

%====================== ACKNOWLEDGEMENTS ======================
\newpage
\pagestyle{plain}
\addcontentsline{toc}{section}{Acknowledgments}
\onecolumn % Single-column.
\if@twoside\else\raggedbottom\fi % Ragged bottom unless twoside option.
\setlength{\footskip}{8mm}
\begin{center}
{
  \large \bf Acknowledgments\\ \vskip 1em
}
\vskip 1em
\end{center}
\singlespace
\doublespace
\hspace{8.5mm}
\vspace{-1em}

% Write your touching message here..

%====================== ABSTRACT ======================
\newpage
\pagestyle{plain}
\addcontentsline{toc}{section}{Abstract}
\onecolumn % Single-column.
\if@twoside\else\raggedbottom\fi % Ragged bottom unless twoside option.

\setlength{\footskip}{8mm}

\begin{center}
{\large \bf Abstract \\ \vskip 1em}
\vskip 1em
\end{center}
\singlespace
\doublespace
\hspace{8.5mm}
\vspace{-1em}

Common law is a legal system where the precedent cases are used to make
judicial decisions. Among many countries like America follow this legal system. Legal
disputes such as in debt collection matter, the Court analyse to similar cases to take decisions
in the case at hand. The court will apply different standards used in other similar cases and
make conclusions using those in the new cases. Therefore to understand the logical structure
of these law, it is necessary to understand each case that has given rise to such law. In this
study, argumentation theory is used to analyse and understand the logical structure of cases
in typical financial legal dispute of the US.

\setlength{\parskip}{0pt} 
%======================= Table of Contents =========================
\newpage
\addcontentsline{toc}{section}{Table of Contents}
\tableofcontents

% Page 2 begins

%===================== List of Figures ======================
\newpage
\addcontentsline{toc}{section}{List of Figures}
\listoffigures

%\clearpage     % \clearpage ends the page, and also dumps out all floats.
%\end{document} % Floats are things like tables and figures.

%\clearpage     % \clearpage ends the page, and also dumps out all floats.

% %===================== List of Tables ======================
\newpage
\addcontentsline{toc}{section}{List of Tables}
\listoftables

% %\clearpage     % \clearpage ends the page, and also dumps out all floats.
% %\end{document} % Floats are things like tables and figures.

%\clearpage     % \clearpage ends the page, and also dumps out all floats.

\setlength{\parskip}{12pt}

