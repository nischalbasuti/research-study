\setlength{\footskip}{8mm}

\chapter{Literature Review} 
\label{ch:literature-review}

% \section{Common Law}

% Common law evolved in England and then adopted by many countries including USA. The main
% difference between common law systems and  other legal systems is that the common law system is based on judgments in case laws that set precedence for new disputes that arise. (Pejovic,2001). One advantage of common law is that it develops gradually overtime, meaning that the law is always adjusted according to the
% situation prevailing in a particular time. In contrast, the common law does not seek to consider the
% future. The court makes decisions only for a particular case at hand considering the specific situation
% at the time.

\section{Marine Insurance}

Marine insurance is an indemnity policy under which an insurer agrees to compensate for losses or damages in consideration of the timely payment of premium. The contract of marine insurance shall cover the clause for indemnity as in no case Assured shall be allowed to make profits out of claim amount. ( The Marine Insurance Act, 1906, Sec. 1 )

% A contract of marine insurance is a contract whereby the insurer undertakes to indemnify the assured, in manner and to the extent thereby agree, against marine losses, that is to say, the losses incident to a marine adventure. ( The Marine Insurance Act, 1906, Sec. 1 )

% Risk is defined as ( The Marine Insurance Act, 1906, Sec. 2 ):

% 1. A contract of marine insurance may by its express terms, or by usage of trade, be extended so as to protect the assured against losses on inland waters or on any land risk which may be incidental to any sea voyage.
% 2. Where a ship is in course of building, or the launch of a ship or any adventure analogous to a marine adventure, is covered by a policy in the form of a marine policy, the provisions of this Act, in so far as applicable shall apply thereto, but, except as by this section provided nothing in this Act shall alter or affect any rule of law applicable to any contract of insurance other than a contract of marine insurance as by this Act defined

% 'Marine perils' means the perils consequent on, or incidental to, the navigation of the sea

Although Marine Insurance is a contract of indemnity, in practice the extent and amount on indemnity are matters of agreement between the parties ( Goole and Hull Steam Towing Co. ltd v. Ocean Marine Ins. Co. Ltd. (1927); Irving v. Manning (1847) ). 

\section{Insurance Policy}

The policy is the basic instrument in a contract of marine insurance ( Section 22 of the Marine Insurance Act, 1906 ).

Subject to the provisions of a statute, a contract of marine insurance is inadmissible in evidence unless it is embodied in a marine policy in accordance with this Act. The policy may be executed and issued either at the time when the contact is concluded or afterward. 

\section{General Average}

The loss to be  borne in common by all the interests involved, and sometimes to denote the contribution to be paid by each separate party involved, in proportion to his interest.

\section{Particular Average}

A particular average loss is a partial loss of the subject-matter insured, caused by a peril insured against, and which is not a general average loss. ( Sec. 64(1) of The Marine Insurance Act, 1906 )

A particular average is a loss borne wholly by the party upon whose property it takes place.
%, and is so called in distinction from a general average for which divers parties contribute ( Phillips (Sec. 1422) ) 

\section{Insurable Interest}

Insurable interest exists when an insured person derives a financial or other kind of benefit from the continuous existence, without repairment or damage, of the insured object.

\section{Perils of the seas}

Perils of the seas are perils that are caused "of the seas", not just "on the seas" ( Wilson, Sons, \& Co. v Xanoth (Cargo Owners) (1887) ).

The term includes losses which are of an extraordinary nature or arise  from some irresistible force which cannot be guarded against by the ordinary exertions of human skill and prudence. It is something which is accidental, unexpected, or fortuitous and not something which is normal, ordinary, and usual in nature.

\textbf{fortuitous event:} An event of natural or human origin that could not have been reasonably foreseen or expected and is out of the control of the persons concerned (as parties to a contract)

\section{Barratry}

The term "barratry" includes every wrongful act wilfully committed by the master or crew to the prejudice of the owner or the charter ( Marine Insurance Act, 1906 )

\section{Proximate Cause of Loss}

Proximate cause is concerned with how the actual loss or damage happened to the insured party and whether it is a result of an insured peril. It looks for what is the reason behind the loss, is that is an insured peril or not.

The fundamental principle underlying every contract of marine peril insured against, or as the legal maxim states: "ausea proxima non remota spectatur" == the proximate and not the remote cause must be looked to. (Section 55 of Marine Insurance Act, 1960)

\section{Canon of Construction}

% % TODO: rewrite in own words.

% Canon of construction refers to a rule used in construing legal documents. It is a means used commonly by the courts to determine legislative intent. A rule of construction serves no good purpose when there is nothing to construe. However, courts should not interpret that which has no need of interpretation, and, when the words of a statute have a definite and precise meaning. Statutes and contracts are read and understood according to the natural language, without resorting to forced construction for the purpose either of limiting or extending their operation. [United States v. Swift, 188 F. 92 (D. Ill. 1911)].

% ---

Judges face different challenges when interpreting the terms of a contract. As a result, different canons exist to aid a court in resolving a dispute between the parties to a contract.

In a contract dispute the court gives contract terms their plain and ordinary meaning, interpreting them as ordinary, average, or reasonable persons would understand them. If the language of the contract is clear and unambiguous, there is no room for further interpretation and the court will enforce the contract as written. By doing so, the court gives effect to the parties' intentions in making the contract and avoids adding its own interpretation to the agreement.


\subsection{Contra proferentem canon}
If the contract contains ambiguous terms, however, they are strictly construed against the party who drafted the contract. This rule of Strict Construction is often applied in contracts containing exculpatory clauses, or provisions that attempt to insulate a party, usually the party who drafted the contract, from liability.

(If there is any doubt about the meaning or scope of an exclusion clause, the ambiguity should be resolved against the party seeking to rely on the exclusion clause. It is the other party who is given the benefit of the doubt.)


\section{The War Risk Policy}

War risk insurance is a type of insurance which covers damage due to acts of war, including invasion, insurrection, rebellion and hijacking. Some policies also cover damage due to weapons of mass destruction. It is most commonly used in the shipping and aviation industries.

The war risk policy includes a clause which excepts from the coverage of the policy “any claim arising from capture, seizure, arrest, restraint, detainment, preemption, confiscation or requisition” by the country in which the vessel is owned or registered.

\section{The Hull Policy}

Hull insurance is an insurance policy especially designed for covering ship damage expenses. Where the 'Hull' refers to the main body of the ship.

\section{Free of Capture \& Seizure Clause}

A clause in ocean marine policies that essentially functions to delete war risk coverage from hull insurance. Losses excluded are those due to nuclear weapons, mines, torpedoes, war (including civil war), piracy, and confiscation or nationalization of property.

\section{Causa proxima, non remota spectator}

Causa proxima, non remota spectator is a Latin phrase which literally translates into ‘the immediate and not the remote cause are to be considered.’ Whenever the cause of any act or circumstance is need to be understood the immediate cause needs to be looked at and not the remote cause. This maxim of causation is applicable for both marine and general insurance. It is the law which gives emphasis to the immediate cause and not the remote occurrence of events.

\section{Argumentation Framework}

% \section{An axiomatic analysis of structured argumentation with priorities}

% \section{Modelling last-act attempted crime in criminal law}

\FloatBarrier

