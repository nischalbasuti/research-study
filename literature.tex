\setlength{\footskip}{8mm}

\chapter{LITERATURE REVIEW} 
\label{ch:literature-review}


\section{Marine Insurance}

From the marine insurance act, 1906:

\begin{quote}
"Marine insurance is an indemnity policy under which an insurer agrees to compensate for losses or damages in consideration of the timely payment of premium. The contract of marine insurance shall cover the clause for indemnity as in no case Assured shall be allowed to make profits out of claim amount." 
\end{quote}\cite{marineInsuranceAct55} 


Although Marine Insurance is a contract of indemnity, in practice the extent and amount on indemnity are matters of agreement between the parties 
\cite{irvinvmanning} \cite{goolevocean}. 

\section{Insurance Policy}

The policy is the basic instrument in a marine insurance contract  \cite{marineInsuranceAct55}.

Subject to the provisions of a statute, a contract of marine insurance is inadmissible in evidence unless it is embodied in a marine policy in accordance with this Act. The policy may be executed and issued either at the time when the contact is concluded or afterward. 

\section{General Average}

The loss to be  borne in common by all the interests involved, and sometimes to denote the contribution to be paid by each separate party involved, in proportion to his interest. \cite{marineInsuranceAct55}

\section{Particular Average}

A partial loss is a particular average  of the subject-matter which is insured, if the loss is due to a peril which it is insured against and is not a general average loss.
 \cite{marineInsuranceAct55}

A particular average is a loss borne wholly by the party upon whose property it takes place.

\section{Insurable Interest}

When an insured person gets some financial or any other type of benefit from  the continued existence an insured object without being damaged, the person is said to have an insurable interest in the said object.

\section{Perils of the seas}

Perils of the seas are perils that are caused "of the seas", not just "on the seas". \cite{wilsonsonsvxanoth}


Losses of extraordinary nature, which arise from a force which cannot be protected from by any normal exertions of human prudence and skill. 
The term includes losses which are of an extraordinary nature or arise  from some irresistible force which cannot be guarded against by the ordinary exertions of human skill and prudence. It is an event which is unexpected, fortuitous or accidental,

\section{Barratry}

It is any wrongful act purposefully done by a master or the crew of a ship which would cause harm, loss or damage to the owner or the charter.
\cite{marineInsuranceAct55}

\section{Proximate Cause of Loss}

Proximate cause deals with how the loss/damage happened to the insured and if it is a result of a peril that is insured . Proximate cause tries to find the reason for the loss and if that is peril that is insured .

The importance of proximate cause in  marine insurance is the legal maxim which states: "Causea proxima non remota spectatur" i.e. the proximate and not the remote cause must be looked to. \cite{marineInsuranceAct55}


\subsection{Causa proxima, non remota spectator}

It is a Latin phrase which translates to ‘the immediate and not the remote cause are to be considered.’

When the court needs to find the event or circumstance which needs to be understood, the immediate cause should be looked at, and not a remote cause. It is the law which emphasizes the immediate cause instead of events that occur remotely.

\section{Canon of Construction}

Interpreting terms of a contract is not a trivial job. To help the court interpret them, there exists some canons to for resolving disputes about a contracts between parties.

If no ambiguity is found in the contract, the contract will be enforced by the court as is written in plain and simple language. By doing so, the court gives effect to the parties' intentions in making the contract and avoids adding its own interpretation to the agreement.


\subsection{Contra proferentem canon}
If the contract contains ambiguous terms, the terms are construed strictly  against the party who has written the contract. This rule kind of "Strict Construction" is often applied in contracts which contain exemption clauses, or means that try to protect a party from liability, which is usually the party that have drafted the contract,

\section{The War Risk Policy}

This type of insurance is used to protect from damages that may occur due to acts of war.

The war risk policy includes a clause which excepts from the coverage of the policy “any claim arising from capture, seizure, arrest, restraint, detainment, preemption, confiscation or requisition” by the country in which the vessel is owned or registered.

\section{The Hull Policy}

Hull insurance is an insurance policy especially designed for covering ship damage expenses. Where the 'Hull' refers to the main body of the ship.

\section{Argumentation Framework}

An abstract argumentation framework \textbf{\textit{AF}} \cite{dung1995}, is defined as: \\
\\
\textit{\textbf{ AF = $<$AR, attacks$>$}}\\
\\
where \textbf{\textit{AR}} is a pair of set of arguments, and   \textit{\textbf{attacks}} is a binary relation on \textbf{\textit{AR}} (i.e  \textit{\textbf{ attacks $\subseteq$ AR $\times$ AR}}). \\

We say that an argument A attacks C if \textit{attacks(A,C)} holds.

 A set S of arguments attacks C if there is an argument in set S that attacks C.

 
 A set S is said to be \textbf{conflict-free}, if there are no arguments A and B in S such that A attacks B. 


 An argument A $\in$ \textbf{\textit{AR}} is \textbf{acceptable} with respect to a set S of arguments iff for each argument B $\in$ \textbf{\textit{AR}}: if B attacks A then B is attacked by S.
 

 A conflict-free set of arguments S is \textbf{admissible} iff each argument in S is acceptable with respect to S.

 A \textit{preferred extension} of an argument framework \textbf{\textit{AF}} is a maximal (with respect to set inclusion) admissible set of \textbf{\textit{AF}}. When someone prefer to accept as many argument as reasonably possible, then it is called preferred extension.

 A conflict-free set of arguments S is called a \textit{stable extension} iff S attacks each argument which does not belong to S. An admissible set S of arguments is called a \textit{complete extension} iff each argument, which is accepted with respect to S, belongs to S. A grounded extension is the least (with respect to set inclusion) complete extension. It is obvious that stable extensions are preferred but not vice versa.

\FloatBarrier

