\setlength{\footskip}{8mm}

\chapter{Introduction}

\section{Background}

Common law is a legal system where the source for judicial decisions are the decisions made by judges in previous cases. In addition to this, there is a legislature that passes new laws and statutes. The relationships between statutes and judicial decisions can be complex. In some jurisdictions, such statutes may overrule judicial decisions or codify the topic covered by several contradictory or ambiguous decisions. In some jurisdictions, judicial decisions may decide whether the jurisdiction's constitution allowed a particular statute or statutory provision to be made or what meaning is contained within the statutory provisions.
Among many countries, United States of America is one which follows Common law.

% TODO: write stuff relevant to insurance instead of finance.
% U.S. legal disputes in Financial
% matters like illegal debt collection or misrepresentation in debt collection letter has been seen more
% frequently for last few decades. While reviewing cases in such financial matters, the court look back
% on similar precedent cases to analyse the different standards that was applied in order to resolved
% the dispute. Each case might give many different conclusions but not all the conclusions are applied by court for the making decisions. Often different versions of a standard are used by different
% courts. For example, the ”Hypothetical Unsophisticated Consumer” and the ”Least Sophisticated
% Consumer”. Both these standard are almost the same. The purpose of both the standard is to protect the consumers who don’t have enough knowledge or understanding of finance. Precedent cases
% in debt collection matters were using the ”least sophisticated consumer” standard but now in recent
% cases the ”Hypothetical unsophisticated consumer” standard is used more. Therefore,

To understand the logical structure of the law in the common law system the only way is to understand the logical structure of the cases that gave rise to that particular law.

\section{Problem Statement}

Common law is the type of law in which the law is derived from the conclusions of previous similar
kinds of cases and tribunals. In order to understand the logical structure of these laws the only way
is to understand the logical structure of similar kind of cases one by one. One way which might
help in understanding the logical structure of the legal disputes and how they come to a conclusion
is by applying argumentation theory. Therefore, the problem considered in this study is to find a
way in which Argumentation theory can be applied to analyze and understand the legal disputes of a
particular case.

\section{Objectives}

% Application of argumentation theory to fit the legal arguments into formal arguments in a typical
% financial legal dispute cases of U.S., one legal case at a time, in order to analyze and understand the
% logical structure of the case, the conclusions supported by the case and how the case is resolved.

Analyzing court cases related to marine insurance where the primary dispute is about proximate cause of loss and modeling the legal arguments in said court cases as formal arguments and applying argumentation theory to understand the logical structure and judgments passed out on each case.

\section{Limitations and Scope}

All the legal cases taken as case study are from marine insurance legal disputes in the United States of America. This study covers the legal disputes which arise due to disagreements on the proximate cause of loss.

\FloatBarrier
